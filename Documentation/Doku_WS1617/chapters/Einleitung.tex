\chapter{Einleitung}
Dieser Projektbericht beschreibt die Tätigkeiten der Autoren im Laufe des 
\ac{HSP} im Wintersemester 2016/2017. Diese beinhalten im wesentlichen 
Folgendes:
Die Ersetzung der kompletten Hardwarearchitektur des ALF und dessen Raspberry Pi 
auf eine neuere, verbesserste Hardwarearchitektur, um mehr Leistungsreserven zu 
besitzen.

\paragraph{Motivation für das Ersetzen des Raspberry Pi} 
\begin{itemize}
 \item Leistungsreserven: Da die alte Hardwareplattform des Raspberry Pi keine ausreichende Performace für zusätzliche Anwendungen, wie zum Beispiel SLAM Algorithmen, besitzt, wurde die Entscheidung getroffen, eine komplett neue Hardware zu erstellen.
\item Um eine möglichst hohe Flexibilität zu erreichen, wurde dabei auf ein FPGA gesetzt. Somit ist es einfach möglich neue Anwendungen hinzuzufügen, oder bestehende Anwendungen zu erweitern.
\end{itemize}

\section{Lesehinweise}
Dieses Dokument ist in mehrere Kapitel gegliedert. Bevor dieses Dokument gelesen und verstanden werden kann folgen hier einige Lesehinweise:
\begin{itemize}
	\item Coderepository - Der gesamte Code und alle relevante Dokumenation zu dem Projekt befindet sich aktuell auf Github. Der Link zum aktuellen Stand ist \href{https://github.com/Alabamajack/Garfield}{https://github.com/Alabamajack/Garfield}. 
	\item Weitere Dateien - Leider beschränkt Github die maximale Dateigröße auf 100MB. Aus diesem Grund liegen Dateien, die größer als 100MB sind, auf dem Laborlaufwerk unter dem Verzeichnis \todo{Pfad zum Laborlaufwerk}. Diese Daten werden im Dokument extra gennant.
	\item Pfade - Alle Dateipfade, die im Dokument genannt werden und für die keine weiteren Informationen angegeben sind, beziehen sich auf das root-Verzeichnis des Coderepositories. Weitere Pfade die verwendet werden sind:
	\begin{itemize}
		\item Pfad im Linux Kernel: Diese Pfade sind absolute Pfade innerhalb einer bestimmten Version des Linux Kernels (nur Releases, keine Pre-Releases etc.). Solche Pfade haben den Prefix \textit{LINUX\_VX.X} wobei \textit{X.X} die Version des Linux Kernel bezeichnet, der verwendet wurde.
		\item Pfade auf dem \ac{HPS} System - Dies sind Linux Distributionspfade. Alle verwendeten Pfade haben als root-Verzeichnis das \textit{home}-Verzeichnis des Standardbenutzers \textit{ubuntu}. Als Prefix dafür wird \textit{HPS} genannt. Sollte ein übergeordneter Pfad zum \textit{home}-Verzeichnis bezeichnet werden ist der Prefix \textit{HPS\_boot}.
	\end{itemize}
\end{itemize}