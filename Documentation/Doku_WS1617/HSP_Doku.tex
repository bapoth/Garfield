\documentclass[a4paper,11pt,bibliography=totoc,listof=totoc]{scrreprt}
\usepackage[utf8]{inputenc}
\usepackage[T1]{fontenc}
\usepackage[ngerman]{babel}


\usepackage{geometry}
\geometry{a4paper}
% ===========================================================
% Optionale Pakete und Einstellungen
% ===========================================================

 \usepackage{scrhack}    % Sollte mit listings zusammen aktiviert werden
 \usepackage{listings}   % Für Code-Beispiele
\usepackage{caption}
 \usepackage{multicol}   % Mehrspaltige Aufzählungen
 \usepackage{amssymb}    % Mathematische Symbole
 \usepackage{amsmath}    % Mathematisch Formeln
\usepackage{textcomp}   % Weitere Symbole
\usepackage{todonotes}
% \usepackage{ifsym}      % Weitere Symbole
\usepackage{url}          % URL Formatierung
\usepackage{dirtree}	% Directory Tree
\usepackage{chngcntr}
\usepackage{longtable}
\usepackage{blindtext}
\usepackage{lscape}
\usepackage{rotating}
\usepackage{afterpage}
\usepackage{graphicx}
\usepackage{longtable}
\usepackage[automark]{scrlayer-scrpage}
\pagestyle{headings}
\counterwithout{footnote}{chapter}
\bibliographystyle{natdin}

\captionsetup{format=hang,margin=10pt,font=small,labelfont=bf}


% \setcounter{tocdepth}{3}   % Tiefe der Überschriften die ins Inhaltsverzeichnis sollen
% \renewcommand{\baselinestretch}{1.5}\normalsize   % Zeilenabstand ändern, falls gewollt
\newcommand{\highlight}{\textbf}   % Zum hervorheben wichtiger Punkte \highlight{} (default: Fett) benutzen (Bei Änderungen nur hier und nicht im ganzen Dokument)
% newcommand für spezielle Begriffe
\newcommand{\kursiv}[1]{\textit{#1}}
% kommando für TODOS:
\newcommand{\TODO}[1]{\todo[inline]{#1}}

% ===========================================================
% Einfügen von PDF's und Bildern
% ===========================================================

%\usepackage[pdftex]{color}
\usepackage{graphicx}

\usepackage{pdfpages}

% ===========================================================
% Literaturverzeichnis
% Muss in Alphabetischer Reihenfolge sein
% ===========================================================

\usepackage[style=numeric,backend=biber, sorting=nyt,natbib=true]{biblatex} % Alternative Zitierung mit [x]
%\usepackage[citestyle=authoryear,bibstyle=authortitle,backend=biber, sorting=nyt,natbib=true]{biblatex} % Ähnlich
%Harvard-Citation style mit \autocite (übernommen aus "Hinweise zum Praktikumsbericht")
%\usepackage[babel,german=guillemets]{csquotes}
\bibliography{Literaturverzeichnis}
% ===========================================================
% Änderung der Überschrift des Literaturverzeichnisses
% ===========================================================

\DefineBibliographyStrings{ngerman}{
    bibliography = {Literaturverzeichnis}
}

% ===========================================================
% Für Abkürzungen und Abkürzungsverzeichnis
% ===========================================================

% option printonlyused - einfügen vor abgaben
\usepackage[footnote]{acronym}

% ===========================================================
% Inhaltsverzeichns
% ===========================================================

\usepackage[colorlinks=true,linkcolor=black]{hyperref}
\usepackage[all]{hypcap}


%====================================================
% Informationen
%===================================================
\date{\today}
\author{Philipp Eidenschink, Florian Laufenböck, Tobias Schwindl\\Matrikelnummern : 3080919, 2894759, 3080498}
\title{HSP Projektbericht}
\def \Projectname{Garfield}
% ===========================================================
% Dokument Anfang
% ===========================================================

\begin{document}

% ===========================================================
% Abstand zwischen Literatureinträgen (muss hier stehen)
% ===========================================================

\setlength{\bibitemsep}{12pt}

% ===========================================================
% NUR ZU TESTZWECKEN!! UNGENUTZTE LITERATUR DARF NICHT VORKOMMEN AM ENDE! Danach kommentieren!!!!!!!!!!!!!! <<<<<<<<<<<<<<<<<<<<<<<<<<<<
%  \nocite{*} % Ganze Literaturliste anzeigen, auch ungenutzte
% ===========================================================

% ===========================================================
% Deckblatt
% ===========================================================

\maketitle
% ===========================================================
% Kurzzusammenfassung - auskommentiert
% ===========================================================
%\begin{abstract}
%\begin{center} \textbf{\abstractname} \end{center} \vspace{\baselineskip}
%Blabla Praktikum, blabla 5. Semester, blabla ...
%\end{abstract}

% ===========================================================
% Inhaltsverzeichnis
% ===========================================================

\tableofcontents

% ===========================================================
% Bericht Teile - Anpassen je nach Notwendigkeit
% ===========================================================
\listoftodos

\chapter{Einleitung}
Dieser Projektbericht beschreibt die Tätigkeiten der Autoren im Laufe des 
\ac{HSP} im Wintersemester 2016/2017. Diese beinhalten im wesentlichen 
Folgendes:
Die Ersetzung der kompletten Hardwarearchitektur des ALF und dessen Raspberry Pi 
auf eine neuere, verbesserste Hardwarearchitektur, um mehr Leistungsreserven zu 
besitzen.

\paragraph{Motivation für das Ersetzen des Raspberry Pi} 
\begin{itemize}
 \item Leistungsreserven
 \item Flexibilität
 \item weils geht und cool ist! 
\end{itemize}


\section{Eingesetzte Tools}
Im folgenden Abschnitt soll ein kurzer Überblick über die verschiedenen eingesetzten Tools gegeben werden. Diese Beschreibung ist nicht vollständig da Grundkentnisse (wie kompilieren eines Linux Kernels aus den Sourcen) vorausgesetzt werden.

\subsection{\ac{FPGA} Entwicklung}

\subsubsection{Quartus}
Das Tool Quartus bildet die Grundlage um für Altera (bzw. inzwischen Intel) \acp{FPGA} entwickeln zu können. Im Prinzip ist es eine Sammlung von verschiedenen Tools, die über eine GUI gesteuert werden. Alle relevanten Prozesse (Building, Generieren von IP-Cores, Systemanalyze etc.) sind auch (u.U. sogar mächtiger) als Kommandozeilentools verfügbar. Das vollständige Handbuch ist unter \href{https://www.altera.com/en_US/pdfs/literature/hb/qts/qts-qps-handbook.pdf}{Quartus Handbuch} zu finden (Achtung: 1939 Seiten!, und das ist nur der erste Teil). Als Überblick und um mit dem Garfield Projekt zu starten gibt es einige nützliche Links und Tutorials wie z.B. \href{https://www.altera.com/support/training/university/materials-tutorials.html}{Altera University Programm - Start}. Eine weitere, sehr empfehlenswerte Anlaufstelle bei Problemen oder auf der Suche nach Application Notes, Tutorials, HowTos und auch Vorträgen ist \href{rocketboards.org}{rocketboards.org}. Dabei handelt es sich um die offizielle Open-Source Sammlung rund um Intel \acp{FPGA}.\\

Die OTH hat eine Reihe von Lizenzen für die gesamte Altera Toolchain (Quartus, SoC EDS, IP-Cores etc.). Wird das Garfield Projekt mit der unlenzierten Version syntethisiert, kompiliert Quartus automatisch einen \textquotedblleft Ablaufzeitstempel\textquotedblright mit ins Design. Der NIOS2 Prozessor und einige \ac{IP}-Cores sind damit nur ca. 30 Minuten lauffähig! Die Lizenzen verwaltet Herr Altmann. Dieser hat auch mind. einen WLAN-USB Stick an dessen MAC-Addresse die Lizenz gebunden ist. Es ist nicht empfehlenswert die Lizenz an eine feste MAC-Addresse eines privaten PCs zu binden, da diese Lizenz dann für andere Studenten verbraucht wäre. Die Installation von Quartus sollte im Hochschulnetz erfolgen, da die Downloadgröße ca. 20GB beträgt.

\subsubsection{Qsys}
Bei QSYS handelt es sich um Intels System Integrations Tool. Es ist eine sehr abstrakte Variante sich ein komplettes \ac{FPGA} System zusammenzuklicken und automatisch jegliche Hardwarebeschreibungen, evtl. notwendige Treiber für den NIOS2 etc. zu generieren. Nach der generierung entsteht ein großer IP-Core, der abschließend in die eigene Pinbeschreibung eingebettett werden muss. Auch für dieses Tool kann am besten wieder auf ein Tutorial \href{https://www.altera.com/content/dam/altera-www/global/en_US/pdfs/literature/tt/tt_qsys_intro.pdf}{QSYS Tutorial} oder auf \href{rocketboards.org}{rocketboards.org} hingewiesen werden.
Da das Garfield Projekt auch zwei selbstgeschriebenen Hardwarekomponenten beinhaltet, müssen die Pfade für diese QSYS bekanntgemacht werden. Dies ist nicht nur notwendig für die Person, die das Hardwaredesign anpasst/syntethisiert, sondern auch für Beteiligte, die Code für den NIOS2 schreiben wollen und auf die \ac{HAL}-Generierung des Tools angewiesen sind. Dazu müssen in QSYS die Einstellungen wie in Abbildung \ref{Settings:IP} dargestellt. Bei geöffnetem QSYS muss unter \texttt{Tools->Options->IP Search Path}.

\begin{figure}
	\includegraphics[width=\textwidth]{Abb/Qsys_settings.png}
	\caption{Einstellungen für die selbstgeschriebenen \ac{IP}-Cores}
	\label{Settings:IP}
\end{figure}

\subsubsection{Quartus Programmer}
Der Quartus Programmer dient dazu, entweder das \ac{FPGA} direkt mit der entsprechenden Image Datei (\texttt{*.sof}-Endung) oder das angeschlossene Flash mit einer \texttt{*.jic} Datei zu flashen. In den entsprechenden Quartus Tutorials sind auch kleine HowTos enthalten, wie mit dem Programmer umzugehen ist. Auf der für das Board zugeschnittene \href{http://www.terasic.com/downloads/cd-rom/de0-nano-soc/}{System-CD} befindet sich auch eine \texttt{DE0-Nano-SoC\_User\_manual.pdf}. In diesem ist auch beschrieben, wie man eine Datei für den Flash erstellt und dies auf den Flash lädt.
\chapter{Architektur}
Hier wird die Architektur beschrieben, die dem ganzen Projekt zugrunde liegt.

 \chapter{Operating System - FreeRTOS}
 Dieses Kapitel beschreibt die Benutzung des verwendeten Betriebssystemes, das FreeRTOS. 
 \section{Tasks}
 \begin{itemize}
  \item Scheduler
  \item Erzeugung
  \item Zyklisch aufrufen
 \end{itemize}

 \section{RTOS Config}
 Hier wird die genaue RTOS Konfiguration beschrieben, die benutzt wurde für das OS auf dem NIOS2.
 \section{benutzte Tasks}
 Hier werden dann die tatsächlich benutzten Task erläutert, die auf dem NIOS2 laufen und für alle wichtigen Aufgaben zuständig sind.

\chapter{Eclipse NIOS2}
\section{C++ - Beschränkungen, besondere Einstellungen}
Das in der 16.1 verwendeten Quartus mit dem mitgeliefertem GCC Compiler hat einige Einschränkungen bezüglich der Verwendung von einigen C++ Features. Alle in den C++ Standardbibliotheken vorhandenen STL Container, z. B. std::vector, std::stack, std::map usw., sind nicht benutzbar. Außerdem ist es nicht möglich die std::string Klasse zu benutzen. Die Benutzung solcher Features führt dazu, dass der Speicher nicht mehr ausreicht. Für die Verwendung dieser Klassen sind auf dem NIOS2 mit der aktuellen Toolchain ca 700KB RAM nötig, allerdings sind nur 128KB RAM vorhanden. Dies wurde vom ALTERA Support Team direkt bestätigt mit der Angabe, dass die Benutzung von C++ im Moment nicht effizient möglich ist und externer Speicher für die Verwendung von C++ angeraten ist. Alle anderen Features hingegen sind, soweit bekannt, ohne Einschränkungen benutzbar.

Für die Unterstützung von c++11 ist eine kleine manuelle Anpassung des Makefiles nötig, welches die Toolchain mit Erstellung eines BSP automatisch generiert. In der Sektion
\begin{itemize}
 \item \# Arguments only for the C++ compiler.
\end{itemize}
ist die Ergänzung folgenden Flags nötig: -std=c++11; da diese Einstellung über die GUI nicht erhalten bleibt. Die Sektion sollte dann folgendermaßen aussehen:
\begin{itemize}
 \item \# Arguments only for the C++ compiler.\\APP\_CXXFLAGS := \$(ALT\_CXXFLAGS) \$(CXXFLAGS) \ \\-std=c++11
\end{itemize}
Diese Einstellung ist auch unbedingt nötig, da einige c++11 Features verwendet wurden und demzufolge ohne diesem Flag die Applikation nicht erfolgreich kompiliert.

\section{Die externen/internen ips}
Qsys bemühen, die mitzunehmen, sonst fehlt da was; siehe git doku?


\subsection{NIOS2 - Treiber / Hardware Abstraction}
Hier wird kurz der Aufbau der HAL bzw. der Aufbau der jetzt zur Verfügung stehenden Treiber des NIOS2 Cores erläutert. Genauere Dokumentation ist in der Doxygen Dokumentation zu finden, die im Anhang mitgeliefert wird. 
\subsubsection{Display}
Das Display welches in diesem Projekt verwendet wurde, besitzt neben dem graphischen Display einen Mikro SD Karten Slot und einen Touchscreen, welche über SPI angesprochen werden. Letztere werden zum aktuellen Zeitpunkt vom Display Treiber nicht unterstützt. Der Dispaly Treiber basiert auf der \href{https://github.com/adafruit/Adafruit-GFX-Library}{Adafruit GFX Library} und läuft zum aktuellen Zeitpunkt mit der maximalen Geschwindigkeit von 24 MHz. Bevor das Display benutzt werden kann, muss es initialisiert werden. Im Anschluss lässt sich eine Zeile auf das Display schreiben. Der Treiber fügt an den Beginn der Zeile die aktuelle Zeilennummer hinzu. Ist das Display bereits durch vorherige Zeilen gefüllt, so wird automatisch von oben neu begonnen. Der Methode zum Schreiben einer Zeile muss somit lediglich die Farbe und Größe übergeben werden.
\subsubsection{Motor}
Die Klasse Drive ist für die Ansteuerung des Motors zuständig. Dabei ist es möglich die maximale Geschwindigkeit im Bereich zwischen 0\% und 100\% zu begrenzen. Eine weitere Funktion ermöglicht das Setzen der Richtung und der Geschwindigkeit, das dann in ein PWM Signal für den Motor umgerechnet wird. 
\subsubsection{Lenkung}
Die Lenkung des Fahrzeuges wird mithilfe der Klasse Steering ermöglicht. Der verwendete Servo Blue Bird BMS-630MG besitzt einen maximalen Winkel von $\pm$60°, welcher sich durch die Init() Funktion begrenzen lässt. Der Servo arbeitet mit einer PWM Frequenz von 125 Hz und erreicht seine maximale Winkel bei einer Periodendauer von 900 bzw. 2100 $\mu$s (Siehe Datenblatt im Repository für weitere Details). Die Funktion Set() setzt dann den tatsächlichen Winkel.
\subsubsection{MPU6050}
Das mpu6050 Modul ist über den IIC Bus angeschlossen und kann die 3 Beschleunigungsachsen, 3 Drehachsen und die Temperatur auslesen. Das Modul muss vor der Verwendung einmalig initalisiert werden. Dabei ist zu beachten, dass der AD0 Pin die IIC Adresse des mpu6050 Bausteines hardwaremäßig verändert. 
\subsubsection{Ultraschall}
Die vier zur Verfügung stehenden Ultraschallsensoren sind über den gleichen IIC Bus, wie das MPU6050 Modul angebunden. Diese Geräte benötigen keine Initialisierung, d.h. sind sofort nach dem Anschließen an den Bus einsatzbereit. Allerdings ist zu beachten, dass nur Geräte an den Bus angeschlossen werden, die unterschiedliche IIC Adressen haben (Funktion zur Änderung der IIC Adresse ist vorhanden), da es sonst zu undefiniertem Verhalten auf dem Bus kommt. Für das ändern der Address sollte nur ein Gerät angeschlossen sein.


\chapter{Zusammenfassung}
Das Projekt konnte erfolgreich abgeschlossen werden.


% ===========================================================
% Abbildungsverzeichnis
% ===========================================================

\listoffigures

% ===========================================================
% Abkürzungsverzeichnis
% ===========================================================

% Muss von Hand sortiert werden!!
% TODO: sortieren mit sort über Kommandozeile
\addchap{Abkürzungsverzeichnis} %Abkürzungsverzeichnis
\markboth{Abkürzungsverzeichnis}{}
\begin{acronym}[LANGER] 			% längste Abkürzung steht in eckigen Klammern
	\acro{ALF}{Autonomes Laser Fahrzeug}
	\acro{HAL}{Hardware Abstraction Layer}
	\acro{HSP}{Hauptseminar Projektstudium}
	\acro{Lidar}{Light detection and ranging}
	\acro{ROS}{Robot Operating System}
	\acro{RVIZ}{ROS Visualization}
	\acro{RISC}{Reduced Instruction Set Computer}
	\acro{SoC}{System-on-a-Chip}
	\acro{FPGA}{Field Programmable Gate Array}
	\acro{IP}{Intellectual Property}
	\acro{HPS}{Hard Processor System}
	\acro{PLL}{Phase-locked loop}
\end{acronym}


% ===========================================================
% Literaturverzeichnis
% ===========================================================

\printbibliography

\appendix
%\chapter*{Anhang}
\markboth{Anhang}{}
\addcontentsline{toc}{chapter}{Anhang} 
\newpage
\renewcommand{\thesection}{\Alph{section}}
\section{}
\dirtree{%
.1 /.
.2 Datasheets\DTcomment{Datenblätter zur verwendeten Hardware}.
.2 Documentation\DTcomment{Dokumentation des Projekts inkl. Schaltplan und Protokoll}.
.2 FPGA\_Design\DTcomment{Projektverzeichnis der FPGA Beschreibung}.
.3 Datasheets\DTcomment{Datenblätter zum verwendeten FPGA}.
.3 Garfield\_Design\DTcomment{Quartus Projekt-/ Konfigurationsdateien und QSYS-Projekt}.
.3 ip\_extern\DTcomment{Verwendete externe IP-Cores}.
.3 ip\_intern\DTcomment{Verwendete interne, selbstentwickelte IP-Cores}.
.3 output\_files\DTcomment{FPGA Images mit Konfigurationsdateien}.
.2 Software\DTcomment{Software Projektverzeichnis}.
.3 common\DTcomment{Gemeinsame Softwarebestandteile getrennt nach Verwendung}.
.3 Software\_ARM\DTcomment{ARM Software - Linux, Comm\_Gateway und alf\_urg}.
.3 Software\_HQ\DTcomment{HQ Software - Garfield Control und melmac\_rviz}.
.3 Software\_NIOS2\DTcomment{NIOS2 Software - FreeRTOS inkl. Treiber}.
}

\end{document}
